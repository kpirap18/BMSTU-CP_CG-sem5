\chapter*{Заключение}
\addcontentsline{toc}{chapter}{Заключение}

В рамках данного курсового проекта были:% проанализированы и рассмотрены существующие алгоритмы трассировки лучей, а также предложен и реализован способ его улучшения. Выбор варианта улучшения был сделан на основе особенностей сцены данного программного обеспечения.

\begin{itemize}
	\item описаны структуры трехмерной сцены, включая объекты, из которых состоит сцена;
	\item проанализированы и выбраны необходимые существующие алгоритмы для построения сцены;
	\item проанализированы и выбраны варианты оптимизации ранее выбранного алгоритма удаления невидимых линий;
	\item реализованы выбранные алгоритмы;
	\item разработано программное обеспечение, которое позволит отобразить трехмерную сцену и визуализировать удар молнии;
	\item проведены сравнение стандартного и реализованного оптимизированного алгоритма удаления невидимых линий.
\end{itemize}
 

%Также в ходе выполнения поставленных задач были получены знания в области компьютерной графики, описаны структуры представления объектов на сцене, построена сцена с двумя объектами (дом и молния) и проведено сравнение оптимизированного и неоптимизированного алгоритма.

В ходе выполнения эксперимента было установлено, что верное определение направлений лучей может значительно улучшить алгоритм. Было выявлено, что отрисовка сцены с помощью улучшенного алгоритма обратной трассировки лучей работает быстрее, чем с помощью стандартного. 

