\chapter*{Введение}
\addcontentsline{toc}{chapter}{Введение}

В современном мире компьютерная графика используется достаточно широко. Типичная область ее применения – это кинематография и компьютерные игры.

На сегодняшний день большое внимание уделяется алгоритмам получения реалистичного изображения. Такие алгоритмы являются одними из самых затратных по времени, потому что они должны учитывать множество физических явлений, таких как преломление, отражение, рассеивание света. Для повышения реалистичности изображения также учитывается дифракция, вторичное, троичное отражение света, поглощение. 

Можно заметить, что чем более качественным является изображение на выходе алгоритма, тем больше времени и памяти используется для его синтеза. Это и становится проблемой при создании динамической сцены, так как на каждом временном интервале необходимо производить расчеты заново. 

Целью данной курсовой работы является создание реалистичной сцены, визуализирующей такое природное явление как гроза, сопровождающаяся вспышками молний. Явление будет продемонстрировано на примере дома, освещенного вспышкой молнии.

Чтобы достигнуть поставленной цели, требуется решить следующие задачи:

\begin{itemize}
	\item описать структуру трехмерной сцены, включая объекты, из которых она состоит;
    \item проанализировать существующие алгоритмы построения изображения и обосновать выбор тех из них, которые в наибольшей степени подходят для решения поставленной задачи;
    \item проанализировать и выбрать варианты оптимизации ранее выбранного алгоритма удаления невидимых линий;
    \item реализовать выбранные алгоритмы;
    \item разработать программное обеспечение для отображения сцены и визуализации удара молнии; 
    \item провести сравнение стандартного и реализованного оптимизированного алгоритма удаления невидимых линий.
\end{itemize}
