\chapter*{Введение}
\addcontentsline{toc}{chapter}{Введение}

В современном мире компьютерная графика используется достаточно широко. Типичная область ее применения – это кинематография и компьютерные игры.

На сегодняшний день большое внимание уделяется алгоритмам получения реалистичного изображения. Такие алгоритмы являются одними из самых затратных по времени, потому что они должны предусматривать множество физических явлений, таких как преломление, отражение, рассеивание света. Для создания еще более реалистичного изображения также учитывается дифракция, вторичное, троичное отражение света, поглощение. 

Можно заметить, что чем качественнее мы получаем изображение на выходе алгоритма, тем больше времени и памяти мы используем для синтеза. Это и становится проблемой при создании динамической сцены, так как на каждом временном интервале необходимо производить расчеты заново. 


Чтобы достигнуть поставленной цели, требуется решить следующие задачи:

\begin{itemize}
    \item проанализировать алгоритм трассировки лучей, чтобы понять какую часть вычислений стоит заменить на квантовые;
    \item проанализировать и сконструировать выбрать квантовые алгоритмы и структуры данных, которые возможно использовать в алгоритме трассировки;
    \item реализовать выбранные квантовые алгоритмы;
    \item провести сравнение рассматриваемых алгоритмов с использованием квантовых и традиционных вычислений.
\end{itemize}
