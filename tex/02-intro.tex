\chapter*{Введение}
\addcontentsline{toc}{chapter}{Введение}

В современном мире компьютерная графика используется достаточно широко. Типичная область ее применения – это кинематография и компьютерные игры.

На сегодняшний день большое внимание уделяется алгоритмам получения реалистичного изображения. Такие алгоритмы являются одними из самых затратных по времени, потому что они должны предусматривать множество физических явлений, таких как преломление, отражение, рассеивание света. Для создания еще более реалистичного изображения также учитывается дифракция, вторичное, троичное отражение света, поглощение. 

Можно заметить, что чем качественнее мы получаем изображение на выходе алгоритма, тем больше времени и памяти мы используем для синтеза. Это и становится проблемой при создании динамической сцены, так как на каждом временном интервале необходимо производить расчеты заново. 

Целью данной курсовой работы является создание реалистичной сцены, на которой присутствуют дом и молния, с использованием выбранных алгоритмов, и при этом оптимизировать данные алгоритмы.

Чтобы достигнуть поставленной цели, требуется решить следующие задачи:

\begin{itemize}
	\item описать структуру трехмерной сцены, включая объекты, из которых состоит сцена;
    \item проанализировать и выбрать необходимые существующие алгоритмы для построения сцены;
    \item проанализировать и выбрать варианты оптимизаций ранее выбранных алгоритмов (если есть необходимоть);
    \item реализовать выбранные квантовые алгоритмы;
    \item разработать программное обеспечение, которое позволит отобразить трехмерную сцену и визуализировать удар молнии;
    \item провести сравнение реализованных оптимизированных алгоритмов и неоптимизированных.
\end{itemize}
